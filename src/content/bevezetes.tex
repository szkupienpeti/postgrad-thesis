\chapter{\bevezetes} \label{ch:bevezetes}

Az emberiség fejlődésének történetében megannyi vívmány tekinthető mérföldkőnek. Ha a kellően távoli múltba tekintünk, nagyon különböző mérföldköveket találunk, mint pl. a tűz használata, a kerék feltalálása, a könyvnyomtatás, vagy éppen a gőzgép feltalálása. Noha egyik jelentőségéhez sem férhet kétség, mégis szinte lehetetlen összehasonlítani őket, annyira különböző területeken hoztak áttörést.

Az elmúlt évtizedekben azonban a technológiai fejlődés drámaian felgyorsult: a mérföldkövek már nem évszázadonként, hanem évtizedenként, vagy akár csupán néhány évenként követik egymást, témájukban pedig egyre inkább az informatika köré összpontosulnak. A számítógépek megjelenésétől datált időszak – amit nevezhetünk a digitalizáció korának vagy a negyedik ipari forradalomnak is – számos innovációjáról már most biztosan kijelenthető, hogy valóban megváltoztatta az életünket, ilyen pl. a személyi számítógép (PC), az internet, és az okostelefonok elterjedése. Rengeteg olyan is van azonban, amelyek ugyan hasonló változásokat ígértek, a mából még nem lehet eldönteni, valóban be is váltják-e majd ezeket a várakozásokat, mint pl. a virtuális valóság vagy éppen a blokklánc.

Napjainkban a\textit{ mesterséges intelligencia} (MI) az az újdonság, ami lázban tartja a világot (ez nem puszta személyes megfigyelés, a vonatkozó cégek részvényárfolyamainak szárnyalása elég erős indikátor). Ezen belül is a \textit{generatív mesterséges intelligencia} (GMI) a leglátványosabb, hiszen új tartalmakat képes létrehozni. Legyen szó akár szövegről, akár képről, akár videóról, ezek a gombnyomásra, akár ingyenesen létrejövő tartalmak jellegüket tekintve ugyanolyanok, mint azok, amiket eddig kizárólag emberek állítottak elő.
Míg a korábbi innovációk döntően csupán az emberi erőt cserélték gépi erőre, vagy a monoton, repetitív feladatokat automatizálták, a generatív mesterséges intelligenciával \textit{látszólag} intellektuális, kreatív vagy akár művészi folyamatokban is lecserélhetővé válik az ember. Adódik tehát a kérdés, hogy hogyan hat ennek a technológiának a megjelenése olyan emberi tevékenységekre, amelyek esetében korábban fel sem merülhetett, hogy automatizáljuk.

Ebbe a sorba tartozik a szoftverfejlesztés is, ami éppen az az intellektuális tevékenység, ami segítségével eddig a többi repetitív folyamatot automatizáltuk. Hiába szoftverfejlesztés eredményei maguk a generatív mesterséges intelligencia szoftverek is, azok nemcsak célként, hanem eszközként is szolgálhatnak a szoftverfejlesztésben, felvetve a kérdést, hogy hogyan is változtatja meg a generatív mesterséges intelligencia magát a szoftverfejlesztési folyamatot.

Ezt illetően szélsőséges véleményekkel találkozhatunk. Míg egyesek szerint a mesterséges intelligencia megjelenése (és különösen az általa generált kódok) miatt csak még nagyobb szükség lesz szoftverfejlesztőkre, mások szerint a nem is olyan távoli jövőben a mesterséges intelligencia már a szoftverfejlesztők munkáját is el fogja venni, hiszen olcsóbban fog jobb kódokat generálni.

Mivel mérnökinformatikusként magam is szoftverfejlesztőként dolgozom, különösen foglalkoztat a kérdés. Az elmúlt években az az érzés alakult ki bennem, hogy nem lehet nem találkozni ezzel a technológiával, mindenki erről beszél, és öngerjesztő módon mindenki attól fél, hogy kimarad belőle. Az új, látványos eszközök mögött ugyanakkor gyakran nem látszik kristálytisztán a valódi hozzáadott érték, így könnyen megkérdőjelezhető, megalapozottak-e a témát övező hatalmas várakozások. Ennek a kérdésnek a megválaszolása (már ha egyáltalán lehetséges) természetesen jóval túlmutat egy szakdolgozat keretein, mégis egyértelmű volt számomra, hogy ezt a témát szeretném körüljárni.

Szakdolgozatomban azt vizsgálom, hogyan hat a generatív mesterséges intelligencia a szoftverfejlesztésre. A \textit{Software Development Lifecycle} (SDLC) fázisain keresztül elemzem, az egyes fázisokban mi lehet a szerepe ennek az új technológiának, valamint, hogy ehhez képest hol tart a gyakorlatban az alkalmazása. Külön figyelmet fordítok azokra a körülményekre, amelyek relevánsak lehetnek a technológia alkalmazhatóságát illetően, ideértve a potenciális nehézségeket, amelyek meggátolhatják az elméleti felhasználási lehetőségek gyakorlati megvalósulását.

Munkám több ponton is kapcsolódik az Informatikai menedzsment posztgraduális képzés tanmenetéhez. A \textit{Szervezeti információrendszerek} kurzuson külön előadás foglalkozott a mesterséges intelligencia hatásaival (\textit{„Mesterséges intelligencia: Revolution vagy Hype?”}), míg a szoftverfejlesztés a \textit{Software engineering} tárgynak volt témája. A mesterséges intelligencia jogi és biztonsági aspektusait az \textit{Infokommunikációs jog} és az \textit{Informatikai biztonság} kurzusok tárgyalták.

A dolgozat \ref{ch:hatterismeretek}. fejezetében áttekintem a téma releváns elméleti ismereteit, a szoftverfejlesztést és a generatív mesterséges intelligenciát, valamint a kettő találkozását, illetve röviden áttekintem a szoftverfejlesztés iparágát. A \ref{ch:kutatasmodszertan}. fejezetben vázolom kutatásom módszertanát, ideértve a kutatási kérdéseket és az adatgyűjtés módját. A \ref{ch:eredmenyek}. fejezetben kutatási kérdésenként részletesen ismertetem az eredményeket. Az \ref{ch:kovetkeztetesek}. fejezetben következtetéseket vonok le a kutatási eredményekből, valamint megfogalmazok egy akciótervet a technológia bevezetésére. Végül a \ref{ch:osszegzes}. fejezetben összegzem a dolgozatot.
