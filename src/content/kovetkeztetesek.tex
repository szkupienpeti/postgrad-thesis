\chapter{Következtetések}\label{ch:kovetkeztetesek}

kb. 5 oldal

\section{A generatív mesterséges intelligencia megítélése}\label{sec:kutmod-ertekeles}

mindenhez ért, de semmiben nem a legjobb, viszont nagyon kényelmes mindenre használni (még arra is, amire lenen jobb (pl. algoritmikus, biztosan helyes) céleszköz) -- akinek kalapács van a kezében, az mindent szögnek néz

meglévő mintákon tanul --> nincs igazi out of the box thinking, igazi kreativitás

hallucináció

bullshitelés (igazi adatok elrejtése, hogy aztán egy másik AI-jal visszafejtsék, pedig nem pontos inverzek)

nem féltik a munkájukat, szükség lesz rájuk (mindegy, hogy konkrétan szoftverfejlesztőnek hívjuk-e, de szükség lesz olyanra, aki valós problémákat digitális megoldásokra tud fordítani, mert ehhez mindkét világot ténylegesen érteni kell)

AI használat fontos, de nem váltja ki az embert. inkább társa/okos asszisztens

absztrahálás képessége (emberi gondolkodás sajátja)

igazi gondolkodás nem spórolható meg

hype

\section{A generatív mesterséges intelligencia szerepe}\label{sec:kutmod-ertekeles}

megengedhető hiba? mennyire kritikus a feladat?

szoftverfejlesztésben jól használható

AI létrehozás veszélyes, mert úgyse nézi át, akinek át kéne

de reviewra, kontextus magyarázatra, nem kritikus dolgokra hasznos és hatékony

kódolás túlreprezentáltság: 

MIT 95\%-ban kudarc: szigetszerűek

\section{A generatív mesterséges intelligencia használatának kihívásai}\label{sec:kutmod-ertekeles}

jogi környezet

vállalati környezet (adatbiztonság, biztonsági megoldások kiherélik a toolokat)

céges tudásbázis integrálás (JIRA, Confluence, kódok, DB-k)

elfogy a token

túlságosan hozzászokunk (elbutulás)

unalmas munka elvégzése helyett unalmas átnézni az AI outputot --> emberi természet, luistaság --> át se nézzük igazán --> veszély (vannak hibák, amiekt elkövetni nem követünk el, de mégsem vesszük észre)

Vince: LLM válasz validálása: ehhez szükséges a tudás formalizálása --> kihívás: emberiség tudását formalizálni (ontológia?)

\section{Akciólista a generatív mesterséges intelligencia bevezetéséhez}\label{sec:akciolista}

jogi környezet tisztázása: mit lehet?

mérhető keretrendszer: ROI, usage, tokenek, költségek

céges tudásbázis integrálása (kockázatok felmérése)

rendszeres dolgozók közti tudásmegosztás, dedikált idő a tanulásra, kísérletezésre

\section{Kutatásmódszertan értékelése}\label{sec:kutmod-ertekeles}

újrafelhasználható módszer a céges AI használat/gondolkodás felmérésére

SDLC jó keretrendszer

diverz csapatok (tapasztalat, beosztás, cégnél töltött évek, cégen belül min dolgoznak)

szeretnek ilyenekről beszélgetni az emberek

AI only, ..., human only besorolás nem működött

panel létszám OK, 1 óra kevés, 1,5 óra OK

nem reprezentatív