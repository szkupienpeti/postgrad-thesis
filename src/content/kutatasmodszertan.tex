\chapter{Kutatásmódszertan} \label{ch:kutatasmodszertan}

Ebben a fejezetben összefoglalom a dolgozatban alkalmazott kutatásmódszertant. A \ref{sec:kutatas-hatter}. alfejezetben vázolom a kutatás hátterét és motivációját, majd a \ref{sec:kutatas-celok}. alfejezetben meghatározom a kutatási célokat. A \ref{sec:kutatas-kerdesek}. alfejezetben megfogalmazom a kutatási célokból következő kutatási kérdéseket, majd a \ref{sec:kutatas-adatgyujtes}. alfejezetben bemutatom az alkalmazott adatgyűjtési módokat. Végül a \ref{sec:kutatas-osszegzes}. alfejezetben összegzem a bemutatott módszertant.

\section{Kutatás háttere}\label{sec:kutatas-hatter}

Az elmúlt évek egyik legforróbb témája a generatív mesterséges intelligencia. Életünk szinte minden részére hatással van, sőt, talán azt sem túlzás mondani, hogy felforgatja azt. A tudományos kutatástól kezdve az orvosláson keresztül az oktatáson át a művészetig sorra nyíltak meg olyan lehetőségek, amelyek korábban elképzelhetetlenek voltak, és ezzel számos korábbi alapvetés is megkérdőjeleződött azt illetően, hogy miben rejlik az emberi tevékenységek emberi mivolta.

Ez a hullám természetesen a szoftverfejlesztést is elérte, sőt, mivel a technológia maga is ide köthető, talán érthető, hogy ezen a területen különösen meghatározó témává vált. A szoftverfejlesztés hatékonyabbá tételének lehetősége joggal hozta lázba a tágan értelmezett szakmát, ami a generatív mesterséges intelligencia eszközök gyors elterjedéséhez vezetett, gyakran alapvetően megváltoztatva ezzel a korábban megszokott munkafolyamatokat.

Mivel magam is szoftverfejlesztőként dolgozom, személyesen is találkozom a téma dilemmáival. Egyik vállalat sem szeretne semmiről sem lemaradni, ezért kifejezetten támogatják ezen eszközök használatát, de ennek a gyakorlatban számos nehézsége is adódik. Ez a személyesen is megtapasztalt kettősség motivált arra, hogy tágabb kontextusban is megvizsgáljam a technológia felhasználását.
Számos tanulmány foglalkozik a generatív mesterséges intelligenciával, azon belül is a szoftverfejlesztésben való felhasználhatóságával. Én azonban ezen belül is azt szeretném megvizsgálni, hogy milyen potenciális ellentmondások húzódnak az elméleti felhasználhatóság és a gyakorlati felhasználás között, vagyis mik azok a körülmények és nehézségek, amelyek éket vernek a potenciális lehetőségek és a rögvalóság közé.

\section{Kutatási célok}\label{sec:kutatas-celok}

Kutatásom célja a generatív mesterséges intelligencia szoftverfejlesztésbeli felhasználásának körüljárása mind elméleti, mind gyakorlati oldalról, majd az elméleti tudás és a gyakorlati tapasztalatok szintetizálása. Mindezt egy olyan rendszerbe foglalva elemzem, amely újrafelhasználható modellként szolgálhat potenciális további kutatások számára is, azaz alkalmas arra, hogy felmérje egy sokaság (pl. egy szervezet) viszonyulását ehhez a technológiához.

Kutatásomat a szoftverfejlesztési életciklus (SDLC) strukturálja, amely egy régóta használt modell a szoftverfejlesztés fázisainak leírására. Az SDLC univerzális fázisain végighaladva elemzem a generatív mesterséges intelligencia felhasználását.

Minden fázis kapcsán áttekintem a szakirodalomban leírt főbb felhasználási lehetőségeket, valamint az iparági és akadémiai interjúalanyok álláspontját a lehetséges felhasználásokról. Összegzem az interjúkon elhangzott, valamint személyesen átélt felhasználási tapasztalatokat, amelyeken keresztül tetten érhető, hol is tart valójában az elméleti lehetőségek gyakorlati megvalósulása. Végül az összegyűjtött tudás és tapasztalat alapján megkísérlem feltárni a gyakorlati megvalósulás mozgatórugóit, szükséges feltételeit, potenciális kihívásait, akár ellehetetlenítő körülményeit.

\section{Kutatási kérdések}\label{sec:kutatas-kerdesek}

Munkám során egyszerre igyekeztem a téma technikai és emberi aspektusait is megvizsgálni. Noha maga a szoftverfejlesztés és a generatív mesterséges intelligencia kifejezetten technikai témák, ezek mellett igyekeztem kidomborítani a szoftverfejlesztésben dolgozók személyes megéléseit is, legyenek azok akár általános gondolatok a témában, akár konkrét felhasználási tapasztalatok. Mindezek alapján az alábbi kutatási kérdéseket fogalmaztam meg, és kerestem rájuk a választ.

\paragraph{RQ1.} Hogyan vélekednek a szoftverfejlesztésben dolgozók általánosságban a generatív mesterséges intelligenciáról?

\paragraph{RQ2.}  Hogyan használható a generatív mesterséges intelligencia a szoftverfejlesztési életciklus (SDLC) különböző fázisaiban?
\begin{enumerate}
	\item Projekttervezés
	\item Elemzés
	\item Rendszertervezés
	\item Fejlesztés
	\item Tesztelés
	\item Bevezetés
	\item Karbantartás
\end{enumerate}

\paragraph{RQ3.} A tapasztalatok alapján mik a generatív mesterséges intelligencia szoftverfejlesztésbeli felhasználásának kihívásai és korlátai?

\section{Adatgyűjtés}\label{sec:kutatas-adatgyujtes}

Kutatási témámat több eltérő nézőpontból is szerettem volna megvizsgálni, így munkám során különböző módon gyűjtött adatokat szintetizáltam. Áttekintettem a releváns szakirodalmat, iparági panelbeszélgetéseket szerveztem, interjúkat készítettem egyetemi oktatókkal, valamint felhasználtam személyes tapasztalataimat is.

\subsection{Szakirodalmi áttekintés}
Kutatásom egyik fő részét a vonatkozó szakirodalom feldolgozása adja. Számos nemzetközi publikációt tekintettem át, többek között konferenciacikkeket, preprint publikációkat, iparági jelentéseket és technológiai leírásokat. Ez egyrészt megalapozta munkám elméleti hátterét, másrészt rávilágított arra is, hogy mennyire újszerű témáról van szó, hiszen számos területen még a tudománynak is csak kérdései vannak.

\subsection{Iparági panelbeszélgetések}
A személyes vélemények és tapasztalatok becsatornázására személyes jelenlétű, kiscsoportos panelbeszélgetéseket szerveztem egy amerikai pénzügyi szolgáltató budapesti irodájának dolgozói között. A sok évtizedes múlttal rendelkező cég működésében a kezdetektől fogva kulcsszerepet játszik a saját fejlesztésű szoftveres megoldásokra épülő innováció, így kiváló terepet szolgáltatott kutatásomhoz. A panelbeszélgetések adatait a \ref{tab:panelek}. táblázat tartalmazza.

\begin{table}[htbp]
	\centering
	\caption{Iparági panelbeszélgetések adatai}
	\label{tab:panelek}
	\begin{tabular}{p{2.4cm} p{1.8cm} p{3.3cm} p{2.2cm} p{2cm}}
		\toprule
		\textbf{Dátum} & \textbf{Időtartam} & \textbf{Résztvevő} & \textbf{Szakmai tapasztalat} & \textbf{Beosztottak száma} \\
		\midrule
		
		\multirow{5}{*}{\textbf{2025. 11. 13.}} 
		& \multirow{5}{*}{90 perc}
		& Szoftverfejlesztő 1.  & 10 év & 0 \\
		& & Szoftverfejlesztő 2. & 20 év & 0 \\
		& & Szoftverfejlesztő 3. & 15 év & 1 \\
		& & Szoftverfejlesztő 4. & 18 év & 2 \\
		& & Szoftverfejlesztő 5. & 18 év & 9 \\
		
		\midrule
		
		\multirow{6}{*}{\textbf{2025. 11. 14.}} 
		& \multirow{6}{*}{75 perc}
		& Szoftverfejlesztő 6.  & 20 év & 0 \\
		& & Szoftverfejlesztő 7.  & 14 év & 0 \\
		& & Szoftverfejlesztő 8.  & 16 év & 0 \\
		& & Szoftverfejlesztő 9.  & 15 év & 1 \\
		& & Szoftverfejlesztő 10. & 32 év & 0 \\
		& & Szoftverfejlesztő 11. & 27 év & 0 \\

\bottomrule
\end{tabular}
\end{table}

A beszélgetéseket rávezetésképp az MI-vel kapcsolatos általános vélemények megvitatásával kezdtük. Ezt követte egy brainstorming a generatív MI szoftverfejlesztésbeli felhasználását illetően, amely során az elhangzott ötletelek cédulákra írtuk, és az SDLC fázisai szerint helyeztük el a táblán. Az ötletelést követően részletesen megvitattuk valamennyi ötletet, konkretizálva, hogy pontosan mit értettek alatta, továbbá kitérve a releváns körülményekre, kihívásokra és korlátokra. Esetenként a múltbeli tapasztalatokon túlmenően arról is kialakultak izgalmas beszélgetések, hogy vajon mit hozhat még a jövő egy-egy felhasználási lehetőséget illetően.

\subsection{Akadémiai interjú}
Az iparágban dolgozók tapasztalatai mellett szerettem volna bevonni az akadémiai szféra meglátásait is, ezért egyetemi oktatókkal is szerettem volna interjúkat készíteni. Az egyetemi nézőpont beemelését több okból is fontosnak tartottam: egyrészt a legmodernebb technológiákat illetően mindenképp releváns a kutatásban élenjáró egyetemi szféra nézőpontja, másrészt a téma a leendő szoftverfejlesztő munkaerőt illetően is felvet kérdéseket, amelyekre előbb-utóbb az egyetemi oktatásnak kell válaszokat adnia.

Végül sajnos csak egy ilyen interjút sikerült elkészítenem. Ezt az interjút kevésbé strukturáltam, mint az iparági panelbeszélgetéseket, hiszen itt kevésbé az SDLC szerinti konkrét tapasztalatokon, inkább az eltérő nézőpont általános meglátásain volt a hangsúly. A Budapesti Műszaki és Gazdaságtudományi Egyetem (BME) Villamosmérnöki és Informatikai Karának (VIK) egyetemi docensével készítettem interjút, amelynek adatait a \ref{tab:egyetemi-interju}. táblázat tartalmazza.

\begin{table}[htbp]
	\centering
	\caption{Akadémiai interjú adatai}
	\label{tab:egyetemi-interju}
	\begin{tabular}{p{2.4cm} p{1.8cm} p{3.3cm} p{2.2cm} p{2cm}}
		\toprule
		\textbf{Dátum} & \textbf{Időtartam} & \textbf{Résztvevő} & \textbf{Szakmai tapasztalat} & \textbf{Tudományos fokozat} \\
		\midrule
		
		\textbf{2025. 11. 20.}
		& 30 perc
		& Egyetemi docens 1. & 14 év & PhD \\
		
		\bottomrule
	\end{tabular}
\end{table}

\subsection{Személyes tapasztalatok}

A kutatásom során saját szakmai tapasztalataimat is felhasználtam. Mivel több éve szoftverfejlesztőként dolgozom nagyvállalati környezetben, közvetlenül találkoztam azokkal a működési sajátosságokkal és mindennapi kihívásokkal, amelyek a szoftverrendszerek fejlesztését és üzemeltetését övezik. A compliance környezet, a komplex örökölt rendszerek, a dokumentáció hiányosságai és az időnyomás mind olyan tényezők, amelyek közvetlenül befolyásolják a generatív mesterséges intelligencia alkalmazhatóságát is. Ezek a tapasztalatok így értékes kontextust adtak ahhoz, hogy a többi adatot értelmezni tudjam.

Az elmúlt években a generatív MI megjelenését nemcsak hírfogyasztó állampolgárként, hanem a munkám során is megtapasztaltam, így első kézből láthattam, milyen elvárások, kérdések és bizonytalanságok merülnek fel a fejlesztői közösségben. Ez a személyes perspektíva természetesen nem helyettesíti a formális adatgyűjtést, de fontos kiegészítő nézőpontot biztosít: segít megérteni, hogyan jelennek meg a technológiai újítások a mindennapi gyakorlatban, és milyen tényezők határozzák meg azok reális alkalmazhatóságát.

\section{Összegzés}\label{sec:kutatas-osszegzes}

A különböző adatgyűjtési módszerek eltérő nézőpontokból világítják meg kutatásom témáját. A dolgozat során a \textit{trianguláció} elvét követve kombináltam a szakirodalmi áttekintést, iparági panelbeszélgetéseket, akadémiai interjút és saját szakmai tapasztalataimat, hogy minél árnyaltabb és megbízhatóbb képet nyújtsak a generatív mesterséges intelligencia szoftverfejlesztésre gyakorolt hatásáról. Az így integrált adatokat a kutatási kérdések mentén rendszerezve és értelmezve elemeztem.

Kutatásom egyik egyértelmű korlátját a válaszadók köre jelenti. Egyrészt a minta mérete csak korlátosan teszi lehetővé általános következtetések levonását, másrészt a résztvevők összetétele sem tekinthető reprezentatívnak a teljes iparágra nézve. Mindezek ellenére az eredmények jól érzékeltetik azokat a trendeket, dilemmákat és tapasztalatokat, amelyek a vizsgált témában jelenleg relevánsak lehetnek. A kutatás így értelmezhető egy kvalitatív feltáró vizsgálatként, amely alapot adhat későbbi, nagyobb mintán végzett kutatásokhoz.

A konkrét megállapításokon túl a dolgozatban alkalmazott strukturált, SDLC-fázisok mentén szervezett megközelítés és a panelbeszélgetéseken alapuló módszertan újrafelhasználható keretet biztosíthat további vizsgálatokhoz. Egy hasonló kutatás nagyobb mintán átfogó képet adhat arról, hogyan gondolkodnak a szoftverfejlesztők pl. egy szervezeten belül a generatív MI lehetőségeiről és korlátairól, illetve hol tartanak annak gyakorlati alkalmazásában.
