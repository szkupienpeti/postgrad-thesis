\chapter{Összegzés}\label{ch:osszegzes}

Szakdolgozatomban azt vizsgáltam, hogyan hat a generatív mesterséges intelligencia a szoftverfejlesztésre. Ehhez elsőként áttekintettem a releváns elméleti ismereteket. Bemutattam a szoftverfejlesztési életciklust, valamint a fejlesztési folyamat legismertebb modelljeit és módszertanait. Áttekintettem a mesterséges intelligencia fejlődésének főbb állomásait, majd a mesterséges intelligencia szoftverfejlesztésre gyakorolt hatásait. Példákkal demonstráltam, hogy már a nemgeneratív MI is komoly hatást gyakorolt a fejlesztési folyamatra, valamint felvillantottam a generatív MI-vel kapcsolatos legfontosabb kihívásokat. Végül röviden áttekintettem a szoftveripar sajátosságait.

Kidolgoztam és bemutattam kutatásom módszertanát. A kutatás hátteréből és motivációjából kiindulva kutatási célokat határoztam meg, majd azokból megfogalmaztam a kutatási kérdéseket. Kutatásom során számos különböző adatgyűjtési módot használtam, amelyeket részletesen bemutattam.

Kutatásom eredményei bemutatták, hogyan ítélik meg a szoftverfejlesztők a generatív mesterséges intelligenciát, mint új technológiát. A szoftverfejlesztési életciklus fázisai mentén haladva részletesen bemutattam, hogy milyen felhasználási potenciál rejlik a technológiában és milyen vonatkozó valós tapasztalatok állnak rendelkezésre. Sorra vettem a felhasználás nehézségeit, elméleti és gyakorlati korlátait.

A kutatás eredményeiből következtetéseket vezettem le. Összegeztem a technológia megítélését, valamint körvonalaztam a szoftverfejlesztésen belüli szerepét. Azonosítottam a felhasználás legfőbb kihívásait, és megfogalmaztam egy rövid akciólistát, amely segítségül szolgálhat a technológia bevezetése előtt álló szervezetek számára. Végül értékeltem az alkalmazott kutatásmódszertant, és kijelöltem a kutatás lehetséges folytatását.